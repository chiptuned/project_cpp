\documentclass[9pts]{article}
\usepackage[utf8]{inputenc}
\usepackage[top=2.5cm, bottom=2.5cm, left=1.5cm, right=1.5cm]{geometry}

\usepackage{listings}
\usepackage{color}

\definecolor{mygreen}{rgb}{0,0.6,0}
\definecolor{mygray}{rgb}{0.5,0.5,0.5}
\definecolor{mymauve}{rgb}{0.58,0,0.82}

\lstset{
  backgroundcolor=\color{white},   % choose the background color; you must add \usepackage{color} or \usepackage{xcolor}; should come as last argument
  basicstyle=\footnotesize,        % the size of the fonts that are used for the code
  breakatwhitespace=false,         % sets if automatic breaks should only happen at whitespace
  breaklines=true,                 % sets automatic line breaking
  captionpos=b,                    % sets the caption-position to bottom
  commentstyle=\color{mygreen},    % comment style
  deletekeywords={...},            % if you want to delete keywords from the given language
  escapeinside={\%*}{*)},          % if you want to add LaTeX within your code
  extendedchars=true,              % lets you use non-ASCII characters; for 8-bits encodings only, does not work with UTF-8
  frame=single,	                   % adds a frame around the code
  keepspaces=true,                 % keeps spaces in text, useful for keeping indentation of code (possibly needs columns=flexible)
  keywordstyle=\color{blue},       % keyword style
  language=Octave,                 % the language of the code
  morekeywords={*,...},            % if you want to add more keywords to the set
  numbers=left,                    % where to put the line-numbers; possible values are (none, left, right)
  numbersep=5pt,                   % how far the line-numbers are from the code
  numberstyle=\tiny\color{mygray}, % the style that is used for the line-numbers
  rulecolor=\color{black},         % if not set, the frame-color may be changed on line-breaks within not-black text (e.g. comments (green here))
  showspaces=false,                % show spaces everywhere adding particular underscores; it overrides 'showstringspaces'
  showstringspaces=false,          % underline spaces within strings only
  showtabs=false,                  % show tabs within strings adding particular underscores
  stepnumber=2,                    % the step between two line-numbers. If it's 1, each line will be numbered
  stringstyle=\color{mymauve},     % string literal style
  tabsize=2,	                   % sets default tabsize to 2 spaces
  title=\lstname                   % show the filename of files included with \lstinputlisting; also try caption instead of title
}

\title{[4AI02] Projet C++: algorithme de dijkstra}
\author{
  Géhère, Kévin\\
  \texttt{kevin.gehere@inserm.fr}
  \and
  Fraillon, Vincent\\
  \texttt{vincent.fraillon@inserm.fr}
}

\begin{document}
\maketitle

\section{Introduction}
Le but de ce projet est de vous montrer de manière sommaire mais réaliste une utilisation réelle de la Standard Template Librairy (\emph{STL}).
Pour ce faire, nous avons choisit de vous faire implémenter l'algorithm de dijkstra,
un algorithme qui vise à évaluer le coût minimal pour aller d'un point A à un point B.
C'est un algorithme célèbre car extrémement utilisé (au moins comme base) pour la planification d'itinéraire.
Un exemple facile serait l'itinéraire proposé par google maps, la ratp ou la sncf pour les transports en commun.\\
C'est algorithme est également une porte d'entrée pour le domaine de la théorie des graphes.

La programmation est dans la plus part des cas simplement un moyen de faire fonctionner une idée simple au travers d'un programme,
ou bien complexe au travers d'une association de programme.
La plus grande difficulté d'un programmation vient donc plus du fait de trouver l'environnement le mieux adapté pour faire fonctionner cette idée que l'idée elle même,
qui est souvent déjà présente et fonctionnelle (cas évident des mathématiques).
Par analogie, il faut poser le sol avant de pouvoir marcher.
Ici nous parlerons d'avantage de structures.

Si vous souhaitez contextualiser le projet vous pouvez vous mettre dans la situation suivante.
Vous êtes membre d'une start-up qui veut se lancer sur le marché de la vente de données du trafic.
Pour ce faire il vous faut implémenter une algorithme capable de planifier des temps de parcours au moyen d'une heuristique fixe ou dynamique.
De plus une base de données réaliste et fiable pour le tester est réquit pour établir la validité de son implémentation.
Dijkstra et la base libre de la RATP sont des choix logiques car robustes.
Après concertation, un contrat est écrit pour juger des fonctionnalités minimums ne fréinant pas l'évolutivité de la solution.
%Vous recevez le cahier des charges des commerciaux qui vous demande les fonctionnalités minimum du projet que vous traduisez en contrat de programmation.

\section{Objectifs}
Un contrat en programmation est le fait qu'une classe hérite d'une classe mère virtuel pure sans implémentation.
Ce principe permet de rajouter des fonctionnalités à une classe sans l'altérer et en garantissant sont appel via un pointeur sur le contrat.
Si la class respecte le contrat, peu importe ses autres fonctionnalités, elle garantit d'avoir implémenter les fonctionnalités du contrat. \\

La class \emph{Generic\_class} aura cet objectif pour vous évaluer (voir annexes).
Vous devrez dériver de cette class pour créer la votre, peut importe le nom que vous lui donnez même s'il serait préférable qu'il soit logique.


\section{Notations}
Dans le but d'évaluer vos connaissances le mieux possible, ce projet sera décomposé en étapes clefs.
Une charte d'écriture vous sera imposé comme dans les entreprises et les projets communs de manière générale.
Celle que nous imposons est un exemple de ce qu'on pourrait exiger mais légérement allégé:\\
\begin{itemize}
\item Compilation en -Wall -Wextra -Werror -O3
\item Utilisation des outils du C++ 11 maximum
\item Proscription de l'utilisation de \emph{new} et \emph{delete}
\item Mémoire dynamique via les containers \emph{STL} ou les outils de \emph{memory} comme \emph{unique\_ptr} (normalement inutile dans le cadre du projet)
\item Indentation parfaite
\item Limitation maximum de duplicat de code
\item Implémentation en snake\_case
\item Tout type créé commence par une majuscule
\item Les constantes déclarées via \emph{define} sont écrites intégralement en majuscules
\item Membres de class appelés via \emph{this}
\item Variables d'entrées précédées d'un underscore \emph{\_}
\item Variables locales neutres
\item Utilisation du mot clef \emph{const} lorsque l'entrée ne doit pas être modifiée
\item Utilisation systématique de référence ou de pointeur lors du passage d'un objet.
\item Respect du typage lors des comparaisons au moyen de \emph{static\_cast$<$Type$>$(var)} si nécessaire
\end{itemize}

Vous avez 4 séances de 4H chacune pour réaliser ce projet.
Lors de chaque séance nous metterons l'accent sur une nouvelle étape à réaliser,
cependant le rythme n'est pas imposé.
Si vous souhaitez aller plus vite ou plus lentement rien ne vous en empêche, car vous pouvez soumettre à tout moment lors de la période du projet.
Des programmes de test vous seront fournit afin d'évaluer votre code et d'hésiter son niveau d'accomplissement.
Les étapes seront:
\begin{itemize}
\item Surcharge des operateurs de flux pour les données et la dérivation du contrat ($+$3pts $=>$ 3$/$20)
\item Lecture \& affichage des stations ($+$4pts $=>$ 7$/$20)
\item Lecture \& affichage des connections ($+$4pts $=>$ 11$/$20)
\item Génération \& affichage du graphe ($+$4pts $=>$ 15$/$20)
\item Estimation du meilleur chemin ($+$6pts $=>$ 22$/$20)
\item (BONUS) heuristique dynamique ($+$6pts $=>$ 28$/$20)
\end{itemize}

Un malus de 4pts maximum sera ensuite appliqué en accord avec la lisibilité et le respect de la charte d'écriture du code.
Tout point au dessus de 20 sera utilisé comme bonus pour les 4 premiers TPs. \\

\appendix
\clearpage
\section{Generic\_class.hpp}
\lstinputlisting[language=C++]{../src/Generic_class.hpp}

\end{document}
